% \documentclass[a4paper,11pt,twoside,class=meetingmins,crop=false,agenda]{standalone}
% \documentclass[a4paper,11pt,twoside,class=meetingmins,crop=false,chair]{standalone}
\documentclass[a4paper,11pt,twoside,class=meetingmins,crop=false]{standalone}
\newcommand{\meetingType}{board}
\usepackage{groupmeetings}

\begin{document}

\setdate{2016--01--19}
\setpresent{
    \chair{A.~Goodsell},
    \boardmember{M.~Fry},
    L.~Foglianti Spadini,
    A.~Hyslop,
    A.~Kallaivannan,
    R.~Kent,
    C.~Lau,
    A.~Rutley,
    S.~Searles-Bryant,
    S.~Wright,
    L.~Yeo
}

\maketitle
% \centering {Malet Place Engineering Building, room 3.14a}

\section{Old business}
% \begin{items}
%     \item Ratification of minutes from 2016-01-13.
% \end{items}
\begin{hiddenitems}
    \item \priormins
\end{hiddenitems}

\section{Announcements \& progress}
\begin{items}
    \item We now have bench space in Lab  3. No access 1300--1400 every day and 1400--1630 on Wednesdays. A 3d printer is available; a sketch must be made and shown to the technician. (RK)
    \item Final report has been started on Overleaf. Team can edit at \url{www.overleaf.com/4075528rppsqb}.
\subsection{Aims and motivations}
    \item Motivation is to build a robot for education. The robot should be programable by children; they set start/end points and waypoints and whether or not to draw in between. This teaches algorithmic thinking (\emph{e.g.} age 6--9).
\subsection{Group structure and roles}
    \item Roles which have been allocated.
    \begin{hiddensubitems}
        \item[(AR, AH)] Art review: Looking into history of sand art; choose type of thing to draw.
        \item[(SW, AK)] Robotics review: Look at movement/digging mech, materials, and history.
        \item[(SSB)] Education review: Case for teaching children algorithmic thinking; existing ways in which this is achieved.
        \item[(AG, AH)] Website: Project website.
    \end{hiddensubitems}

    \item Design outline: Areas of design which should be decided upon this week.
    \begin{itemize}
        \item Movement (SW, AK)
        \item Interface
        \item Guidance (LFS, CL)
        \item Construction materials (SW, AK)
        \item Digging mechanism (SW, AK)
        \item Electronics platform (CL, SSB)
    \end{itemize}
\subsection{Preliminary research}
    \item Art review. (AR, AH)\\
        Began as protest of commercialization of art in US. Motivations behind recent works are to bring attention to the environment.\\
        There are many large scale sculptures using rakes, trowels. Smaller sand patterns use ropes; precise measurement is not emphasized. New movement of small-scale `sand-stories' involving light projection.
    \item History of robotics review (incl. preliminary survey of methods of movement and digging mechanisms). (SW, AK)
    \begin{subitems}
        \item Drawing mechanics: most viable options are rakes/teeth and rollers.
        \item Locomotion: problems include purchase on sand and not ruining the drawing. Tracks ruin previous drawing; best option looks like SandBot: three wheels.
        \item Brief summary of history of robotics
        \item Materials: waterproof/sandproof towels exist, could be used to protect sensitive components.
    \end{subitems}
    \item Education review. (SSB)\\
        Turtles have been used for education since 1960s. Drawing robots exist based on the Logo language; ink-and-paper equivalents of what we aim to build are very popular in primary schools.\\
        The idea of teaching computation in primary schools has received enormous interest recently.
    \item Electronics and guidance mechanisms (LFS, CL)
    \begin{subitems}
        \item GPS and compass: compass necessary to determine direction of robot. Cost would be c.~\pounds{50}. Not sure if precision is adequate for our needs.
        \item Laser grids: prebuilt modules are too expensive; should look into building our own. Ultrasound equivalent is too short-range.
        \item Tethering: requires child to follow the robot.
        \item Remote control (bluetooth): can use a mobile phone; range of 30 feet. Software exists for Arduino.
    \end{subitems}
    \item Electronics platform (SSB)
    \begin{subitems}
        \item Arduino: best option for embedded systems; large community online for support. Cost < \pounds{50}.
        \item NETduino: similar to Arduino but uses .NET framework. Possibly easier to program (several team members know .NET languages).
        \item Raspberry Pi: more a computer than for embedded systems. Used a lot for education.
        \item Micro-controller: Requires all supporting circuitry to be built by us; more difficult to program. Possible but would require a lot of time.
        \item A systems engineer role should be appointed to ensure compatibility of all components (software and hardware). LFS points out that power supplies will be an issue that needs some attention.
    \end{subitems}
\end{items}

\section{New Business}
\begin{items}
    \item Resources and examples (MF)
    \begin{subitems}
        \item Rake: adjustable tines. Too large for our project but mechanism may be of interest
        \item ``Ed-bot'': small line-following robot built by MF's son.
        \item USB shield for Arduino: includes dongle for cordless Playstation controller
        \item Assorted servo motors.
    \end{subitems}
    \item Time management plan and Gantt chart. (AG)
        \begin{subitems}
            \item SSB thinks more troubleshooting/testing time should be added. AG says build phases include time for testing.
            \item First software task needs to be a tool to allow hardware to be tested manually.
            \item A second Gantt chart should be produced once the design has been completed and detailed development processes are known.
        \end{subitems}
    \item Discussion of material to be posted on the website. (AG)
    \begin{subitems}
        \item Demonstration of website.
        \item Mirrors report in sections with summaries of each part.
        \item In future will have photos and a video of the robot.
    \end{subitems}
    \item Peer-assessment scores need to be collected
\end{items}

\vspace{1em}
\nextmeeting{Tuesday, January 26, at 11:15 in MPEB 3.14a}
\vspace{1em}

\section{Actions}
\begin{items}
    \action{Add research to the report on Overleaf}{all}
    \action{Investivate building our own LiDAR system for guidance.}{SSB}
    \action{Collect peer assessment scores}{RK}
\end{items}

\end{document}
