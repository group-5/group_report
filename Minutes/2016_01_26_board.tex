% \documentclass[a4paper,11pt,twoside,class=meetingmins,crop=false,agenda]{standalone}
% \documentclass[a4paper,11pt,twoside,class=meetingmins,crop=false,chair]{standalone}
\documentclass[a4paper,11pt,twoside,class=meetingmins,crop=false]{standalone}
\newcommand{\meetingType}{board}
\usepackage{groupmeetings}

\begin{document}

\setdate{2016--01--26}
\setpresent{
    \chair{A.~Goodsell},
    \boardmember{M.~Fry},
    L.~Foglianti Spadini,
    A.~Hyslop,
    A.~Kallaivannan,
    R.~Kent,
    C.~Lau,
    A.~Rutley,
    S.~Searles-Bryant,
    S.~Wright,
    L.~Yeo
}
%\absent{B.~Gone \textit{(sabbatical)}}
%\alsopresent{\boardmember{P.~Bartlett}}

\maketitle
% \centering {Malet Place Engineering Building, room 3.14a}


\section{Old business}
\begin{items}
    \item \priormins
    \item The report is now available to view in Dropbox at \url{https://goo.gl/1CujQq} or to view and edit in Overleaf at \url{www.overleaf.com/4075528rppsqb}. These will be updated and kept synced with SSB's local copy roughly daily.
    \item Peer assessment form has been filled out for this week
\end{items}


\section{Announcements \& progress}
\begin{items}
    \item We do not have access to our space in Lab 3 on Tuesdays unless we make special arrangements. We also only have half a bench.
    \item Servos: We have 7 in the lab, MF may have more. Plan is to use 3 for the rakes.
    \item Motors: 148:1 gear ratio. We have 4 in the lab from the stair-climbing project.
    \item The code for the Physics study room (top floor of Physics Bldg) is 0921.
\subsection{Electronics platform (SSB, SW, AH)}
    \item The electronics platform outline. Discussion of exact requirements.
\begin{hiddenitems}
    \item We believe that we need the following for the project (from Adafruit in the US):
    \begin{itemize}
        \item Arduino board: Arduino Uno (\${24.95})\\ \url{https://www.adafruit.com/products/50}
        \item Motor control shield (\${19.95})\\ \url{https://www.adafruit.com/products/1438}
        \item GPS shield (\${49.95})\\ \url{https://www.adafruit.com/products/1272}
        \item Servo control shield (\${17.50})\\ \url{https://www.adafruit.com/products/1411}
        \item Compass breakout board (\${19.95})\\ \url{https://www.adafruit.com/products/1746}
    \end{itemize}
    \item The GPS could be a breakout (\url{https://www.adafruit.com/products/746}) instead of a shield at a saving of \${15}.
    \item The servo control board is unnecessary if only 2 servos are used since the motor shield board can handle these.
    \item The motors and servos should be powered separately from the control board. 2 PP3 (9V) batteries should be sufficient.
    \item The total cost for all of the components is a little over \${128}, or \pounds{90}, plus shipping.
    \item We would also need a few other components (bypass capacitors; cables; battery connectors) which could be sourced in the UK (\eg Maplin).
    \item The Arduino (Genuino) 101 has built in accelerometer. This would be very useful for navigation and making the positioning more precise.
    \decisions Procure Genuino 101 from Arduino in Italy. All shields and the compass breakout from Adafruit. RK will order through Lab 3. SSB will forward details to RK and CL.
\end{hiddenitems}



\subsection{Robot design (AR, RK)}
    \item CAD design made.
    \item Use acrylic to manufacture (easy to work with; won't rust)
    \item Could have an acrylic box over top for protection
    \item Dimensions TBC
    \item Discussion of whether two driven wheels should be at the front or the back
    \item AG has a friend who will design a pretty box for us to be 3d-rpinted.
    \item Suggestion from MF: consider using only 1 rake up/down to keep simple
    \item Consider how much we can achieve this year; don't be over ambitious and perhaps leave some development for a further project.
    \item We could recycle wheels from previous project (cable spools).
    \item Safety regulations are extensive.

\subsection{Software (AG, LFS, LY)}
    \item Started looking at Arduino boards but handed that off to electronics subgroup because it made more sense for them to do it.
    \item LY has investigated generating images (fractal-type diagrams) without using GPS. It is feasible to draw images without a positioning system even with possible turning and moving errors accounted for (computer simulations). This would be a good avenue to pursue if GPS and current plans fail. (MF: minimum goal should be to have a spiral-drawing robot tethered to a centre-point.)
    \item LY has found a website for simulating Arduino output for some code (\url{https://123d.circuits.io/}). We can also connect simulated circuits, readings \emph{etc.} for simulation of the whole system. We should start a project on this website to simulate our code.
\end{items}


\section{New Business}
\begin{items}
\item Jobs to be assigned in group general meeting tomorrow.
\end{items}

\vspace{1em}
\nextmeeting{Tuesday, February 2, at 11:15 in MPEB 3.14a}
\vspace{1em}

\section{Actions}
\begin{items}
    \action{Written work to be sent to SSB for compilation}{all}
    \action{Aims and Motivations sections to be written up}{AG}
    \action{Investigate procurements options for shields}{SSB, AH, SW}
    \action{Buy Arduino controller board}{RK, SSB}
    \action{Buy other electronics}{RK, SSB}
\end{items}

\end{document}
