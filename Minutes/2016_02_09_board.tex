% \documentclass[a4paper,11pt,twoside,class=meetingmins,crop=false,agenda]{standalone}
% \documentclass[a4paper,11pt,twoside,class=meetingmins,crop=false,chair]{standalone}
\documentclass[a4paper,11pt,twoside,class=meetingmins,crop=false]{standalone}
\newcommand{\meetingType}{board}
\usepackage{groupmeetings}

\begin{document}

\setdate{2016--02--09}
\setpresent{
    \chair{S.~Searles-Bryant},
    \boardmember{M.~Fry},
    L.~Foglianti Spadini,
    A.~Hyslop,
    A.~Kallaivannan,
    R.~Kent,
    C.~Lau,
    A.~Rutley,
    S.~Wright,
    L.~Yeo
}
\absent{A.~Goodsell \textit{(apologies)}}
% \alsopresent{\boardmember{P.~Bartlett}}

\maketitle
% \centering {Malet Place Engineering Building, room 3.14a}
\section{Old business \& announcements}
\begin{items}
    % \item Ratification of minutes from 2016--02--02.
    \item \priormins
\end{items}

\section{Progress \& discussion}
\begin{items}
    \subsection{Building the robot}
        \item We have all of the components that have been ordered, including a battery for the motors
        \begin{subitems}
            \item MF suggests last year's wheels are too big. Team confirm and plan to cut them down.
            \item The new battery has come; it is surprisingly light (designed for flying robots).\\
            AH concerned the capacity is not enough (20 mins). SSB replies 20 minutes is only at maximum efficiency. Estimates around 1 hour of overall beach drawing time. Also suggests we can use more batteries since they are so light.
            \item Complication: charging the new battery. SSB in conversation with supplier to determine charging pattern (constant current, then constant voltage to finish).
            \item We still need the acrylic for the chassis. AR will source; it will cost \pounds{23} (2/3 days shipping) for a piece that is shaped.
            \item RK shows a picture of new prototype of the robot chassis. AR points out that the wheels need to be narrower; this will affect the design of the drawing mechanism. RK suggests IoM can cut the wheels for us.\\
            MF suggests we should wrap the wheels to improve grip on sand.
            \item AR has a prototype drawing mechanism attached to a servo. Group suggest that two arms per servo should be used.\\
            The method by which the servos will be attached to the chassis needs consideration.
        \end{subitems}
    \subsection{Robot control}
        \item We have put together the Genuino and the motor control shield and we have been able to get the motors moving.
        \item We had trouble with the code for the servo control.
    \subsection{Website}
        \item The website is live at \url{http://group-5.github.io/sand-e/}.
\end{items}

\section{New Business}
\begin{items}
    \subsection{Timeline for building the robot}
        \item The electronics can be assembled inside a box. RK: The box for electronics is in the lab.
        \item A production plan should be written by the group this week.
    \subsection{Mid-term review}
        \item We have been assigned to review Group 6.
        \item SW and LFS have volunteered to go to one of their meetings and interview members of the team for the review.
        \item SSB will contact the group's board member to arrange this.
\end{items}

\vspace{1em}
There will be no board meeting next week because it is reading week.
\nextmeeting{Tuesday, February 23, at 11:15 in MPEB 3.14a}
\vspace{1em}

\section{Actions}
\begin{items}
    \action{Source acrylic for chassis}{AR}
    \action{Give GitHub usernames to SSB or LY for access to website and project report}{all}
    \action{Write production plan for the robot}{all}
    \action{Contact Prof. Ford regarding the mid-term review}{SSB}
\end{items}

\end{document}
