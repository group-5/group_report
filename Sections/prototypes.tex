\chapter{Prototypes and Manufacture}\label{prototypes}\label{section \thechapter}

\mySection{Initial models}{\RK}
    \begin{figure}[bh]%{I}{0.45\textwidth}
        % \vspace{-11pt}
        \centering
        \begin{subfigure}[t]{0.45\textwidth}
            \missingfigure{Cardboard model of \SandE}
            % \includegraphics[width=\textwidth]{Files/carboard_model_1}
            \caption{The underneath of the cardboard model. The wheel that can be seen on the left represents a caster wheel; the other two wheels are attached to motors. The drawing mechanism can be seen on the top. All of this is attached to a base.}
            \label{fig:cardboard model 1}
        \end{subfigure}
        ~
        \begin{subfigure}[t]{0.45\textwidth}
            \missingfigure{Cardboard model of \SandE}
            % \includegraphics[width=\textwidth]{Files/carboard_model_2}
            \caption{The top side of the cardboard model. The larger cube on top of the base is the housing for the electronics. The smaller cube is the battery. The drawing mechanism is the rolled up piece of cardboard on the far left of the base.}
            \label{fig:cardboard model 2}
        \end{subfigure}
        \caption{A cardboard model of \SandE}
        \label{fig:cardboard model}
        % \vspace{-20pt}
    \end{figure}
    It was decided in the ?? week that a model of the robot should be created from cardboard to reveal design problems before a real prototype is created (Figure \ref{fig:cardboard model}). It was also decided that the final robot should fit in a suitcase that could be taken as hand luggage on board a British Airways flight, this robot was designed to be ($56 \times 45 \times 25$) cm. This was done to demonstrate the functionality of the robot to help decide requirements for the final design, as there were several design ideas which each had their own costs and benefits.\\
    The main sections of the robot are the front wheels, back caster wheel, axels, electronics storage, battery, drawing mechanism, base, and motors. All have been represented in this model. Black tape was used as the binding agent for everything; blu tack was used additionally with the wheels. Everything was made from cardboard of two thicknesses. Also, the axels were made from tea stirring thin wooden sticks.

    \begin{figure}[tb]%{I}{0.45\textwidth}
        % \vspace{-11pt}
        \centering
        \begin{subfigure}[t]{0.45\textwidth}
            \missingfigure{Wooden model of \SandE}
            % \includegraphics[width=\textwidth]{Files/wooden_model_1}
            \caption{The top side of the wooden model. The base and drawing mechanism can be seen from this angle. The drawing mechanism is a piece of wood overlapping with the edge of the base, teeth could be attached to finish it.}
            \label{fig:wooden model 1}
        \end{subfigure}
        ~
        \begin{subfigure}[t]{0.45\textwidth}
            \missingfigure{Wooden model of \SandE}
            % \includegraphics[width=\textwidth]{Files/wooden_model_2}
            \caption{The bottom side of the wooden model. The base and wheels can be seen from this angle. The two wheels at the edge of the desk are fixed and the other wheel is a caster wheel.}
            \label{fig:wooden model 2}
        \end{subfigure}
        \caption{A wooden model of \SandE}
        \label{fig:wooden model}
        % \vspace{-20pt}
    \end{figure}
	It was found that this model is too large, and lacked the movement that would make it useful for testing design functionality. A wooden model was then attempted, it was decided that this second model should have movable wheels the same as the cardboard model and the drawing mechanism should be included if possible (Figure \ref{fig:wooden model}).\\
    This design was movable, but provided no useful functionality for testing design specifications and so was quickly finished with.

    \mySection{First Prototype}{\RK}
    \begin{figure}[htb]%{I}{0.45\textwidth}
        % \vspace{-11pt}
        \centering
        \begin{subfigure}[t]{0.45\textwidth}
            \missingfigure{Perspex model of \SandE}
            % \includegraphics[width=\textwidth]{Files/perspex_prototype_1}
            \caption{The first side of the Perspex model. The motors have been fixed onto metal holders which have then been fixed on the base using nuts and bolts. The axels are fixed into the motors and then the axel goes through the middle of the wheels; the wheels are kept in place with a nut either side.}
            \label{fig:wooden model 1}
        \end{subfigure}
        ~
        \begin{subfigure}[t]{0.45\textwidth}
            \missingfigure{Perspex model of \SandE}
            % \includegraphics[width=\textwidth]{Files/perspex_prototype_2}
            \caption{The second side of the Perspex model. Nuts and bolts can be seen protruding through the base. The axels are fixed into the motors and then the axel goes through the middle of the wheels, the wheels are kept in place with a nut either side.}
            \label{fig:wooden model 2}
        \end{subfigure}
        \caption{A Perspex prototype of \SandE}
        \label{fig:perspex prototype}
        % \vspace{-20pt}
    \end{figure}
    After spending several weeks on design the group decided to move on to producing a prototype. Perspex was the chosen material for the base, motors, wheels and axels were obtained from a previous year's project. Holes were drilled into the Perspex and the motors were attached, then the axels and wheels were attached to them (Figure \ref{fig:perspex prototype}).
