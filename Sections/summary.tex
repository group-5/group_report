% A final summary. This should form the basis of the Executive Summary, but this final summary gives you the opportunity to give more supporting detail for someone who wants more - but not a great deal more - information than in the Executive Summary.
% ---from the project guidance

\chapter{Summary}\label{summary,conclusion}\label{section \thechapter}
\glsresetall


\mySection{Background and objectives}{\AG}\label{summary: why}\label{summary: introduction}\todo{title of this section needs thought}
The invention of the difference engine in 1833, came with it an entire new process of thought and development. Since then, the field of computation and robotics has become one of the largest in the world. It is a field that requires developers to suspend their usual way of thinking and instead look at tasks and problems as a computer. These ways of thinking are not encountered in daily life. Other intellectual and physical fields have their foundations discovered through juvenile interactions. However, the ability to approach a task with an algorithmic method---a loop or conditional statements---is not naturally acquired. These skills must therefore be taught through non-conventional means. This could be as a teenager or adult, when first being confronted with programming; however, years of educational studies have shown that skills and understanding are much more easily acquired at an early age. This is the motivation of the \SandE project: to develop a method, useable by schools and parents, to begin to nurture programming skills in children.

The aim for the \SandE project is to develop a working prototype of a programmable, autonomous robot with the ability to draw large scale art in sand. The robot will be able to accept a set of repeatable instructions from a student. These instructions could include: navigation, for example an instruction to go from point A to B before turning and traveling to point C; the ability to control whether the robot should be drawing at any given time, and also the thickness of the line drawn. The student will create these instructions through a graphical user interface, before they are uploaded to the robot and their design is created in sand. The intention is that by passing a design to a robot, students will be forced to think about how a computer will tackle a task they would find simple. Hopefully, students will be able to understand that computers use a system of specific logic that can create wonderful things, which would take much longer by hand.
