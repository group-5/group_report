% In this section:
% Someone should be able to build a robot just like ours from the information in this section.
% Include all drawings, specifications, decisions and processes.

\chapter{Design Specification}\label{design specification}\label{section \thechapter}

% Here: Description of what the robot should do/be

\mySection{Robot design and hardware}{writer TBA}
\towrite{hardware}

    \mySubsection{Drawing Mechanism}{\AK}
        The research done into digging mechanisms (Section 2.2.3) played a central role during the design phase of the mechanism that            would be used on Sand-E. A strong material that would have the strength to be dragged through sand was required, with metal being         the obvious choice. With regards to the part of the mechanism that was in contact with the sand, this was required to be sharp           enough to be able to penetrate the sand to a depth that would leave a sufficient print whilst also not causing too much resistanc         e to be overly restrictive on the motion of the Sand-E.

        Research into digging mechanisms had shown that multiple teeth dragging through the sand was generally accepted as giving the            best results, whether this be through simply using a rake, or having custom built teeth. Consequently three servos would be              mounted onto the back of the robot with two teeth on each individual digging component. This meant Sand-E would have six points          of contact in the sand when trying to leave a print.
        
        When the building the prototype digging mechanism two pieces of metal wire, both 2.5mm diameter were attached to a 5mm wire using         a spot welder. The two pieces of wire that would act as the teeth digging into the sand were then bent using a prong cam so they         would be perpendicular to the direction of motion of the robot. During initial testing of this prototype, by mechanically                dragging it through sand, it was found that the imprint left was too faint. So it was apparent that a new design would be                required with thicker teeth. The prototype however was found to have satisfactory strength so the new version was designed with          thicker wire used for the teeth, and the attachment to the robot. would have a smaller diameter. This was in order to make sure          the weight of the mechanism doesn’t become excessive as this would jeopardize the ability of the servos to raise and lower the           teeth.
        
        For the updated version, the teeth were created using 5mm metal wire. The wire was bent into shape using a bar bander and then           the ends were sharpened using a metal grinder. A thinner piece of 3mm metal wire was used to attach the teeth to the robot. The          ground clearance of the robot was measured and a piece of wire was cut using the guillotine that was long enough to reach the            ground. One end of this piece of wire was filed down to fit into the servo using a lathe. A C02 welder was then used to attach           the two pieces of metal to complete the digging arm. This component performed well in the initial tests so two more versions were         made so the whole digging mechanism was ready to be attached to Sand-E.


\mySection{Electronics and control systems}{writer TBA}
\towrite{electronics}


\mySection{Software and user interface}{writer TBA}
\towrite{software}


\mySection{Other considerations}{writer TBA}
\towrite{other considerations}
