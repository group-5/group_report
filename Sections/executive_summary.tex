\fancyhead[RO]{Executive Summary}
% There is general agreement on the structure of an executive summary - books and training courses emphasize similar points. Typically, an executive summary will:
%
% be approximately 5-10% of the length of the main report
% be written in language appropriate for the target audience
% consist of short, concise paragraphs
% begin with a summary
% be written in the same order as the main report
% only include material present in the main report
% make recommendations
% provide a justification
% have a conclusion
% be readable separately from the main report
% sometimes summarize more than one document
% ---from Wikipedia

% An Executive Summary of not more than 2 A4 pages. This should convey briefly, but precisely, an overall summary of the project, including aims and objectives, the work done, and the final outcome(s). It should be written in such a way as to convince the busy MD of Physics Innovations (UCL) that you have delivered the goods. Remember that in real life, the Executive Summary is often the most important part of a report in leading to the crucial decision. It must, however, be fully backed up by the more extensive detailed material in the body of your report.
% ---from Bartlett's guidance

\chapter*{Executive Summary}\label{executive summary}
\addcontentsline{toc}{chapter}{Executive Summary}
\towrite{Executive summary}

\myBlindSection{Introduction}{writer TBA}


\myBlindSection{Design}{writer TBA}


\myBlindSection{Conclusion}{writer TBA}
