\chapter{Limitations and Future Developments}\label{limitations}\label{futures}\label{section \thechapter}
\section{Future Developments}\label{Limitations and Future Developments: Future Developments}
As  discussed  with  the  project
’
s  board  member,  part  of  the  process  involves  suggesting  and 
discussing  the  extension  of  the  project  for  future  years
’
  teams.  While  the  rest  of  the  report  details 
what  we  have  accomplished  in  the  limited  time  available,  this  section  ai
ms  to  highlight  what  could 
be achieved, building upon this year
’
s project. \par
An  area  for  further  development is  the  guidance  and sensing  system:  there  is
  the  potential 
for an ultrasound sensor to be installed. This would allow an object avoi
dance routine to be written, 
providing  handling  for  obstacles  encountered  (the  limitations  of  such  tec
hnology  are  discussed  in 
this  report).  By  including  such  functionality  in  the  robot,  its  au
tomated  capabilities  would  be 
improved 
–
 this would improve ease-
of
-use, greatly reducing the burden on the user to prepare the, 
potentially large, operating environment. I.e by clearing obstacles.  \par
The GPS data could be applied in a failsafe system. Such a system would pro
vide protection 
against  drowning,  in  the  case  the  robot  was  programmed  to  enter  the  water.  The  ro
bot's  location 
(and  planned  subsequent  location)  could  be  compared  with  maps  of  known  location
s  of  water 
bodies 
–
 an emergency stop (or return to start point) could be triggered if the rob
ot were in danger 
of entering the water.  \par
The original motivation for the robot was to educate children on the b
asics of programming 
–
  to  give  them  a  means  to  use  and  enhance  their  high-level  'thinking' 
skills.  This  would  have  been 
achieved through a user interface on the computer. \par

TO BE COMPLETED
