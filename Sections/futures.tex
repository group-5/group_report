\chapter{Limitations and Future Developments}\label{limitations}\label{futures}\label{section \thechapter}
\section{Future Developments}\label{Limitations and Future Developments: Future Developments}
As  discussed  with  the  project's  board  member,  part  of  the  process  involves  suggesting  and discussing  the  extension  of  the  project  for  future  years' teams.  While  the  rest  of  the  report  details what  we  have  accomplished  in  the  limited  time  available,  this  section  aims  to  highlight  what  could be achieved, building upon this year's project.

An  area  for  further  development is  the  guidance  and sensing  system:  there  is the  potential for an ultrasound sensor to be installed. This would allow an object avoidance routine to be written, providing  handling  for  obstacles  encountered  (the  limitations  of  such  technology  are  discussed  in section~\ref{outline:guidance}).  By  including  such functionality  in  the  robot,  its  automated  capabilities  would  be improved; this would improve ease-of-use, greatly reducing the burden on the user to prepare the potentially large operating environment (\ie by clearing obstacles0.

The \gls{GPS} data could be applied in a failsafe system. Such a system would provide protection against  drowning,  in the  case  the  robot  was  programmed  to  enter  the  water.  The  robot's  location (and  planned  subsequent  location)  could  be  compared  with  maps  of  known  locations  of  water bodies. An emergency stop (or return to start point) could be triggered if the robot were in danger of entering the water.

The original motivation for the robot was to educate children on the basics of programming: to  give  them  a  means  to  use  and  enhance  their  high-level  thinking skills.  This  would  have  been achieved through a user interface on the computer.

TO BE COMPLETED
