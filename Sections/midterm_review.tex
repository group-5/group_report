\chapter{Midterm Review}\label{midterm review}\label{section \thechapter}
\writers{A review was conducted of the group process and team dynamics of Group 6. This review was written by \SW~and \LFS.}

\vspace{15pt}

Based on the interviews and observations conducted on Friday 11th February, the group was found to be cohesive and enthusiastic about their project and overall, in line with the expectations of the interviewers. In particular the group was found to have given itself an internal hierarchy to optimise the efficiency in making decisions without compromising individual preferences or misallocating the different skillsets. This was possible thanks to the election of a team leader, Abetharan.\\
Other relevant roles were assigned to other members to facilitate organisational and logistic issues. The score collection is addressed to Vicky James for her high reliability while the external communication was attributed to Matthew for his communication skills. The duty of writing the report in latex was given to Markos because of his previous experience in scientific paper publishing. Kevin has exemplified the team's resourcefulness, reaching out to relevant departments at UCL for insight into the project.

This particular project involves a large amount of research to be done ahead of committing to a specific path is, as its focus is mainly theoretical. For this purpose, two subgroups were initially instituted, researching how the experiment could have been set up with a mechanical approach and an electronic alternative.\\
From the start the group knew the mechanical approach was more likely to be the best. This likelihood was estimated by the group from previous knowledge of thermodynamics combined with insights provided by the new material explained by the supervisor Ian Ford. In the interest of covering both bases while at the same time providing the strongest backing to the approach most likely to succeed, they divided their resources accordingly; six to the mechanical subgroup and four to the electronic. This resource allocation and workstream subdivision constitutes an efficient application of risk management theory and it is believed to deserve appreciation.

When initial investigations into both paths had been concluded, the case for each was made and the mechanical approach selected -- the electronic team then reintegrated, without friction, into the main workstream. The decision to leave the electronic approach and allocate all the human resources to the mechanical one was taken unanimously and encountered no opposition. This highlights the team's strong ability to communicate their findings, weigh the options, commit to an idea and cooperate on its execution.\\
Small negligence was also found: internal deadlines for the weekly peer assessment were missed and two apologies for absences at formal meetings were forgotten. Although all meetings took place with at least nine of the ten members, a meeting has yet to start on time. It was reported that two members showed little regard for punctuality, repeatedly being late for appointments. These were only prompted on Facebook without obtaining any improvements on this aspect.\\
The team's project timeline was successfully determined ahead of commencing the project; they have succeeded in sticking to this timeline with no modification. This indicates both the quality of the initial plan and the team's ability to keep to their plans.
\subsection*{Recommendations}
To improve the group's productivity and morale it is recommended that all members of the team arrive punctually. It is suggested that the leader takes on the responsibility to enforce this; by prompting them face to face rather than only by the virtual communication already attempted. It might be helpful to increase the meeting frequency; Group 5 has found it useful to establish a weekly informal meeting where all the members are present, along with board meetings. This might serve to improve workload balancing across the team, an area that was highlighted for improvement during interview.
